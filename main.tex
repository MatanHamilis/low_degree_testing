\documentclass[professionalfonts,11pt, aspectratio=169]{beamer}
%Information to be included in the title page:
\usepackage[]{amsfonts}
\usepackage[]{amssymb}
\usepackage[]{amsmath}
\usepackage[]{xspace}
\usepackage{xcolor}
\usepackage{hyperref}
%
% Todo list
\usepackage[colorinlistoftodos,prependcaption,textsize=tiny]{todonotes}
\newcommand{\unsure}[1]{\todo[linecolor=red,backgroundcolor=red!25,bordercolor=red]{#1}}
\newcommand{\change}[1]{\todo[linecolor=blue,backgroundcolor=blue!25,bordercolor=blue]{#1}}
\newcommand{\info}[2][1=]{\todo[linecolor=OliveGreen,backgroundcolor=OliveGreen!25,bordercolor=OliveGreen,#1]{#2}}
\newcommand{\improvement}[2][1=]{\todo[linecolor=Plum,backgroundcolor=Plum!25,bordercolor=Plum,#1]{#2}}
%\newcommand{\thiswillnotshow}[2][1=]{\todo[disable,#1]{#2}}
%
\newcommand{\of}[1]{\ensuremath{{\left(#1\right)}}}
\newcommand{\callig}[1]{\ensuremath{\mathcal{#1}}}
\newcommand{\assign}{\ensuremath{\leftarrow}}
\newcommand{\pr}[1]{\ensuremath{\text{\bf Pr}\brackets{#1}}}
\newcommand{\expectation}[1]{\ensuremath{\mathbb{E}\of{#1}}}
\newcommand{\variance}[1]{\ensuremath{\mathbb{V}\of{#1}}}

\newcommand{\tuple}[1]{\ensuremath{{\left(#1\right)}}}
\newcommand{\braces}[1]{\ensuremath{{\left( #1 \right)}}}
\newcommand{\cbraces}[1]{\ensuremath{{\left\{#1 \right\}}}}
\newcommand{\brackets}[1]{\ensuremath{{\left[ #1 \right]}}}
\newcommand{\floor}[1]{\ensuremath{{\left\lfloor #1 \right\rfloor}}}
\newcommand{\ceil}[1]{\ensuremath{{\left\lceil #1 \right\rceil}}}

\newcommand{\bigo}[1]{\ensuremath{\mathcal{O}\of{#1}}}
\newcommand{\bigomega}[1]{\ensuremath{\Omega\of{#1}}}

\newcommand{\ZerOne}{\ensuremath{\{0,1\}}}
\newcommand{\ZeroOneToThe}[1]{\ensuremath{\{0,1\}^{#1}}}

\newcommand{\ToDo}[1]{{\textcolor{red}{\textbf{TODO:} #1}}}

\newtheorem{proposition}{Proposition}
%\newtheorem{theorem}{Theorem}
%\newtheorem{lemma}{Lemma}
\newtheorem{claim}{Claim}
%\newtheorem{definition}{Definition}

\newcommand{\complexityclass}[1]{\callig{#1}}
\newcommand{\NPclass}{\complexityclass{NP}}
\newcommand{\Pclass}{\complexityclass{P}}

\newcommand{\fcal}{\callig{F}}
\newcommand{\pcal}{\callig{P}}
\usepackage{svg}
\usepackage{tikz}

\usetheme{Frankfurt}
\definecolor{mygreen}{rgb}{.20,.20,.4}
\usecolortheme[named=mygreen]{structure}
%\usecolortheme{default}
% \setbeamertemplate{footline}[page number]{}
\setbeamertemplate{navigation symbols}{}
\setbeamercovered{transparent}
\usefonttheme[onlylarge]{structuresmallcapsserif}

\newcommand{\poly}[2]{\ensuremath{\callig{P}_{#1,#2}}}
\newcommand{\abs}[1]{\ensuremath{\left| #1 \right|}}
\newcommand{\ppr}[2]{\ensuremath{{\textbf{Pr}_{#1}\brackets{#2}}}}
\newcommand{\mfo}{\ensuremath{{\texttt{MFO}}}\xspace}

\AtBeginEnvironment{theorem}{%
  \setbeamercolor{block title}{bg=red!70!black}
}
\AtBeginEnvironment{proof}{%
  \setbeamercolor{block title}{bg=purple!70!black}
}
\AtBeginEnvironment{definition}{%
  \setbeamercolor{block title}{bg=gray!70!black}
}

\newtheorem{observation}{Observation}
\newenvironment<>{examplesecond}[1][]{%
  \setbeamercolor{block title observation}{fg=white,bg=blue!75!black}%
  \begin{observation}#2[#1]}{\end{observation}}



\title{Low Degree Testing}
\subtitle{Seminar in Sublinear Algorithms}
\author[Matan Hamilis]{Matan Hamilis \\ \texttt{matan.hamilis@gmail.com}}
\institute{Reichman University}
%\date{\today}
% \logo{\includesvg[height=1cm]{Reichman_University.svg}}

\begin{document}

\frame{\titlepage}
\begin{frame}[c]
\frametitle{The Plan}
\tableofcontents[pausesections]
\end{frame}

%%%%%
%%%%%
%%%%%
%%%%%

\section{Introduction}

\subsection{Basic Terminology}
\begin{frame}
    \frametitle{Notation}
    \begin{itemize}
        \item \callig{F} denotes some field.
        \item A variable is denoted by a lowercase letter from the end of the alphabet (e.g \(x,y,z\)).
        \item A constant  is denoted by a lowercase letter from the beginning of the alphabet (e.g \(a,b,c\)).
        \item A vector is an overlined lowercase latin letter (e.g \(\bar{a},\bar{x}\)).
        \item \(\bar{x}_i\) is the \(i^{\text{th}}\) coordinate of \(\bar{x}\).
    \end{itemize}
\end{frame}

\begin{frame}
    \frametitle{Univariate Monomials}
    \begin{definition}[Monomial]
        Let \(x\) be a variable, and \(a\) be a non-zero constant and \(d\in \mathbb{N}\).
        The term \(ax^d\) is called a \alert<1>{degree \(d\) (univariate) monomial}.
    \end{definition}
    \pause{}
    \begin{itemize}[<+->]
        \item When the degree is implicit in the context, we use the term ``monomial'' without mentioning the degree.
        \item \(a\) is also called \alert<3->{coefficient}.
    \end{itemize}
\end{frame}

% \begin{frame}
%     \frametitle{Polynomials}
%     \begin{definition}[Univariate Polynomial]
%         Let \(f:\callig{F}\to \callig{F}\) be a non-zero function. 
%         We say \(f\) is a \alert<1->{univariate polynomial of degree \(d\)} if there exist \(a_0,a_1,\ldots,a_d\) such that \(a_d\neq 0\) and
%         \[
%             \forall x \in \callig{F}: f\of{x} = \sum_{i=0}^d a_i x^i
%         \]
%     \end{definition}
%     \pause{}
%     \begin{itemize}
%         \item \(a_0,\ldots a_d\) are also called the \alert<2>{coefficients} of the polynomial.
%     \end{itemize}
% \end{frame}
\begin{frame}
    \frametitle{Polynomials}
    \begin{definition}[Univariate Polynomial (Alt.)]
        Let \(f:\callig{F}\to \callig{F}\) be a non-zero function. 
        We say \(f\) is a \alert<1->{univariate polynomial of degree \(d\)} if it can be expressed as the non-empty sum of monomials of degree at most \(d\).
    \end{definition}
    \pause{}
    \begin{itemize}
        \item Note: We define the zero polynomial to be of degree \(\infty\), so we shall disregard it in this talk.
    \end{itemize}
\end{frame}
\begin{frame}
    \frametitle{Multivariate Monomials}
    \begin{definition}[Monomials]
        Let \(a\) be a non-zero constant, \(x_1,\ldots, x_m\) be \(m\) variables over \(\fcal\) and let \(d_1,\ldots,d_m\in \mathbb{N}\).
        Also let \(d = \sum_i d_i\).
        The product \(a x_1^{d_1}\dotsm x_m^{d_m}\) is called a \alert<1>{total degree \(d\) multivariate monomial}.
    \end{definition}
    \pause{}
    \begin{itemize}
        \item We also call \(d_i\) the \alert<2>{degree of variable} \(x_i\) in the monomial.
    \end{itemize}
    \pause{}
    \begin{example}
        \begin{itemize}
            \item \(x_1x_2\) of total degree 2.
            \item \(x_1^3\) of total degree 3.
            \item \(10\) of total degree 0.
        \end{itemize}
    \end{example}
\end{frame}
\begin{frame}
    \frametitle{Multivariate Polynomials}
    \begin{definition}[(Multivariate) Polynomial]
        Let \(m\in \mathbb{N}\) and let \(f:\callig{F}^m\to \callig{F}\) be a non-zero function. 
        We say \(f\) is a \alert<1->{multivariate polynomial of degree \(d\)} if it can be expressed as a non-empty sum of multivariate monomials of total degree \(\leq d\).
    \end{definition}
    \pause{}
        \begin{example}
         \(f\of{x_1,x_2}=3x_1 + 2x_2^2x_1 + x_2x_1\) is a multivariate polynomial of total degree 3.
        \end{example}
\end{frame}

\subsection{Motivation}
\begin{frame}
    \frametitle{Motivation}
    \begin{itemize}[<+->]
        \item Let \(f:\callig{F}^m \to \callig{F}\) and \(d\in \mathbb{N}\).
        \item Natural Question: Is \(f\) a degree \(\leq d\) polynomial?
        \item Observation: We tackled a weaker form of this problem before. (Where?)
    \end{itemize}
\end{frame}

\begin{frame}
    \frametitle{Linear Functions are Polynomials}
    \begin{itemize}[<+->]
        \item Linear functions are degree-1 polynomials.
        \item Following the BLR theorem, we have built a tester for \alert<2>{linearity}.
        \item A natural extension of linearity testing is \alert<3>{low degree} testing.
        \item Today: Low degree tester for both univariate and multivariate polynomials.
    \end{itemize}
\end{frame}

\section{Background}
\subsection{Polynomials}
\begin{frame}
    \frametitle{Notation}
    \begin{itemize}[<+->]
        \item \(\pcal_{d}\) is the set of univariate polynomials of degree \(\leq d\).
        \item \poly{m}{d} is the set of \(m\)-variate polynomials of degree \(\leq d\).
        \item \(d\) denotes the degree of some polynomial, \(d \leq \frac{\abs{\callig{F}}}{2}\).
        \item \(f:\callig{F}^m \rightarrow \callig{F}\) an \(m\)-variate function.
    \end{itemize}
\end{frame}

\begin{frame}
    \frametitle{Basic Properties}
    Let \(f,g\) be two polynomials of degrees \(d_f,d_g\) respetively.
    \pause{}
    \begin{itemize}[<+->]
        \item The function \(h=f+g\) is a polynomial of degree \(\leq \max\cbraces{d_f,d_g}\)
        \item The function \(h=f\cdot g\) is a polynomial of degree \(d_f+d_g\)
    \end{itemize}
    

\end{frame}
\begin{frame}
    \frametitle{All Functions Are Polynomials}

    \begin{lemma}
        The function \(\pi_0:\callig{F}\rightarrow \callig{F}\):
        \[
                \pi_0\of{x}    =
            \begin{cases}
                1 & x=0 \\
                0 & \text{otherwise}
            \end{cases}
        \]
        is a polynomial of degree at most \(\leq \abs{F}-1\)
    \end{lemma}
\end{frame}

\begin{frame}
    \frametitle{All Functions Are Polynomials}
    \begin{proof}
        \[\pi_0\of{x}=\overbrace{\prod_{a\in \callig{F} \backslash \cbraces{0}}\frac{a-x}{a}}^{\text{\(\abs{\callig{F}}-1\) polys of deg 1}}\]
    \end{proof} 

\end{frame}

\begin{frame}
    \frametitle{All Functions Are Polynomials}
    \begin{lemma}
        For all \(a\in \callig{F}\) the function \(\pi_a:\callig{F}\rightarrow \callig{F}\):
        \[
                \pi_a\of{x}    =
            \begin{cases}
                1 & x=a \\
                0 & \text{otherwise}
            \end{cases}
        \]
        is a polynomial of degree at most \(\leq \abs{F}-1\)
    \end{lemma}
\end{frame}

\begin{frame}
    \frametitle{All Functions Are Polynomials}
    \begin{proof}
        \[\pi_a\of{x}=\pi_0\of{x-a}\]
    \end{proof} 
\end{frame}

\begin{frame}
    \frametitle{All Functions Are Polynomials}

    \begin{theorem}
        All univariate functions \(f:\callig{F} \rightarrow \callig{F}\) can be written as polynomials of degree at most \(\abs{\callig{F}}-1\)
    \end{theorem}
    \pause{}
    \begin{proof}
        \[
            f\of{x} = \sum_{a\in \callig{F}} f\of{a}\cdot \pi_a\of{x}
        \]
    \end{proof}
\end{frame}

\begin{frame}
    \frametitle{All Functions Are Polynomials}

    \begin{theorem}
        All multivariate functions \(f:\callig{F}^m \rightarrow \callig{F}\) can be written as polynomials of degree at most \(m\cdot \braces{\abs{\callig{F}}-1}\)
    \end{theorem}
    \pause{}
    \begin{proof}
        \begin{itemize}
            \item By induction on \(m\). For \(m=1\) we're done.\\
        \pause{}
        \item Assuming the claim holds for \(m-1\) we prove for \(m\):
        \[
            f\of{x_1,\ldots,x_{m-1},x_m} = \sum_{a\in \callig{F}} \overbrace{f\of{x_1,\ldots,x_{m-1},a}}^{\text{deg} \leq \braces{m-1}\cdot  \braces{\abs{\callig{F}}-1}}\cdot \pi_a\of{x}
        \]
        \end{itemize}
    \end{proof}
\end{frame}

\begin{frame}
    \frametitle{Low Degree Polynomials}
    \begin{observation}
        \(\frac{\abs{\callig{F}}-1}{\abs{\callig{F}}}\)-fraction of the functions are of maximal degree.
    \end{observation}
    \pause{}
    \begin{observation}
        Only \(\frac{1}{\abs{\callig{F}}^{\abs{\callig{F}}-1-d}}\)  fraction of the univariate functions are of degree \(\leq d\).
    \end{observation}
    \pause{}
    \begin{definition}[low degree polynomial]
        A function \(f\) is a \alert<1>{low degree polynomial} if it is of degree \(\leq d\).
    \end{definition}
\end{frame}

\begin{frame}
    \frametitle{Interpolation}
    \begin{itemize}[<+->]
        \item Let \(f \in \callig{P}_d\) be a univariate polynomial of degree \(\leq d\).
        \item Question: Given evaluations of \(f\), can we reconstruct \(f\)'s coefficients?
        \item Question: How many such evaluations do we need?
    \end{itemize}
    \uncover<4->{
    \begin{theorem}[Interpolation]
        Let \(f\in \callig{P}_d\) be a univariate polynomial of degree \(\leq d\), then any \(d+1\) samples of \(f\) suffice to reconstruct the \(d+1\) coefficients of \(f\).
    \end{theorem}
    }
\end{frame}
\begin{frame}
    \frametitle{Interpolation}

    \begin{proof}
        Assume we have \(f\of{x_0}\ldots f\of{x_{d}}\) for some distinct set of points \(x_0\ldots x_{d}\) then we can obtain \(f\)'s coefficients \(a_0,\ldots,a_d\) by solving the following linear equation:
\[
    \begin{pmatrix}
        x_0^0 & x_0^1 & \dotsb & x_0^d \\
        x_1^0 & x_1^1 & \dotsb & x_1^d \\
        \vdots & \vdots & \ddots & \vdots \\
        x_d^0 & x_d^1 & \dotsb & x_d^d \\
    \end{pmatrix}
        \begin{pmatrix}
            a_0 \\
            a_1 \\
            \vdots \\
            a_d
        \end{pmatrix}
        =
        \begin{pmatrix}
            f\of{x_0} \\
            f\of{x_1} \\
            \vdots \\
            f\of{x_d} \\
        \end{pmatrix}
\]
    \end{proof}
\end{frame}

\begin{frame}
    \frametitle{Roots of a Univariate Polynomials}
    \only<1-2>{
    \begin{definition}
        Let \(f:\fcal^m \to \fcal \in \pcal_{m,d}\) be a univariate polynomial.
        We say a point \(x \in \fcal^m\) is a \alert{root} of \(f\) if \(f\of{x}=0\).
    \end{definition}
    }
    \pause{}
    \begin{lemma}
        A \alert{univariate} polynomial \(f\in \pcal_d\) has a most \(d\) roots.
    \end{lemma}
    \pause{}
    \only<3->{
    \begin{proof}
        \begin{itemize}[<+->]
            \item Assume \(f\) has \(d+1\) roots, \(x_0,\ldots,x_d\).
            \item Using the linear equation systems we obtain its coefficients:
\[
    \begin{pmatrix}
        x_0^0 & x_0^1 & \dotsb & x_0^d \\
        \vdots & \vdots & \ddots & \vdots \\
        x_d^0 & x_d^1 & \dotsb & x_d^d \\
    \end{pmatrix}
        \begin{pmatrix}
            a_0 \\
            \vdots \\
            a_d
        \end{pmatrix}
        =
        \begin{pmatrix}
            0 \\
            \vdots \\
            0 \\
        \end{pmatrix}
        =
        \begin{pmatrix}
            f\of{x_0} \\
            \vdots \\
            f\of{x_d} \\
        \end{pmatrix}
\]
\item The only solution is \(\bar{0}\), since the matrix is ``Vandermonde'' so it's invertible.
\item Therefore \(f\not \in \pcal_d\) since it's the zero-polynomial, contradiction!
        \end{itemize}

    \end{proof}
    }

    

\end{frame}

\begin{frame}
    \frametitle{Schwartz-Zippel Lemma}
    \begin{itemize}[<+->]
        \item What about number of roots of multivariate polynomial?
        \item The following is given without a proof.
    \begin{theorem}
        Let \(p:\fcal^m \to \fcal\) be a non-zero \(m\)-variate polynomial of total degree \(d\) over finite field \(\fcal\). Then,
        \[
            \ppr{x\in \fcal^m}{p\of{x}=0} \leq \frac{d}{\abs{\fcal}}
        \]
    \end{theorem}
    \end{itemize}


    

\end{frame}
\begin{frame}
    \frametitle{Intermission}
    \begin{itemize}[<+->]
        \item Reminder: Our goal is to build a tester for the property of a function being a low-degree polynomial.
        \item To give some intuition we shall begin with describing a tester for univariate polynomials.
        \item Next, we will present the full multivariate tester.
    \end{itemize}
\end{frame}
\begin{frame}
    \frametitle{Property Testing --- Notation}
    \begin{itemize}[<+->]
        \item Since we present a tester in this talk, we remind a few useful notations.
        \item For functions \(f,f':\fcal^m \to \fcal\) the (normalized) \alert{distance} between \(f,f'\) is defined:
        \[
            \delta\of{f,f'}= \frac{\abs{\cbraces{x:f\of{x}\neq f'\of{x}}}}{\abs{\fcal}^m}
        \]
        \item For property \(\Pi{}\) the \alert{distance} of a function \(f\) from \(\Pi{}\) is defined:
        \[
            \Delta_{\Pi}\of{f} = \underset{f'\in \Pi}{\min} \cbraces{\delta\of{f,f'}}
        \]
    \end{itemize}
    

\end{frame}
\subsection{Intuition}
\begin{frame}
    \frametitle{Intuition}
    \begin{itemize}[<+->]
        \item For the univariate case consider the following tester:
        \begin{enumerate}[<+->]
            \item Samples \(d+2\) points \(x_1,\ldots,x_{d+2}\in\mathcal{F}\)
            \item Interpolates a degree \(d\) polynomial \(p\) from \(x_1,\ldots,x_{d+1}\) and \(f\of{x_1},\ldots,f\of{x_{d+1}}\)
            \item Checks \(p\of{x_{d+2}}=f\of{x_{d+2}}\)
        \end{enumerate}
        \item If \(f\) is low degree, the tester always accepts.
        \item If \(f\) is not low degree then probability of \(x_{d+2}\) being a point of disagreement between \(f\) and \(p\) is at least:
        \[
            \frac{\delta\of{f,p}\cdot \abs{\fcal}}{\abs{\fcal}-\of{d+1}} > \delta\of{f,p} \geq \Delta_{\pcal_d}\of{f} = \epsilon
        \]
    \end{itemize}
\end{frame}
\begin{frame}
    \frametitle{Lower Bound?}

    \begin{theorem}
        No univariate low-degree tester exists which makes less than \(d+2\) queries.
    \end{theorem}
    \begin{proof}
        (sketch, ``Yao's minimax-principle'') \(d+1\) queries convey no information. 
        That is, given \(d+1\) queries' responses we can show that there exist two functions \(f_1,f_2\) that both agree on the responses of the queries, but \(\deg\of{f_1}\leq d\) and \(\deg\of{f_2}=\abs{\fcal}-1\).
    \end{proof}
\end{frame}
\begin{frame}
    \frametitle{Lines}
    \begin{itemize}[<+->]
    \item It can be a bit difficult generalizing the foregoing tester for multivariate setting.
    \item We give an alternative tester that will form the basis for the generalized, multivariate tester.
    \item The alternative tester is based on querying the first \(d+2\) points over a random \alert{line}.
    \item A \emph{line} is parametrized by two values \(r,s\in\fcal\) and is made of the solutions \(y\)to the linear equation \(y=r+s\cdot x\) for \(x\in\fcal\).
    \end{itemize}
\end{frame}

\begin{frame}
    \frametitle{Alternative Tester}
    \begin{itemize}[<+->]
        \item We devise the following tester:
    \begin{enumerate}[<+->]
        \item Sample \(r,s \gets \fcal \) uniformly.
        \item Query the points \(\cbraces{r+s\cdot i}_{i\in\cbraces{0,\ldots,d+1}}\).
        \item Make sure the \(d+2\) points interpolate into a degree \(d\) polynomial.
    \end{enumerate}
    
    \item Obviously, if \(f\) is low degree, the tester accepts.
    \item However, if \(f\) is \(\epsilon\)-far from being low degree, how many random lines will fail the test?
    \item The answer to this question is special case of the following multivariate tester.
    \item It is also the core result of this talk.
    \end{itemize}
\end{frame}

\begin{frame}
    \frametitle{Multivariate Tester}

    \begin{itemize}[<+->]
        \item In the multivariate setting we are given \(f:\fcal^m \to \fcal\).
        \item We will apply the line-based approach.
        \begin{definition}
            A \emph{line} in \(\fcal^m\) is a (\abs{\fcal} long) sequence of the form \(\of{\bar{x} + i \bar{h}}_{i\in \fcal}\) for some \(\bar{x},\bar{h}\in\fcal^m\).
        \end{definition}
        \item It's easy to see that if \(f\) is low degree, then its restriction to any line is a low degree univariate polynomial.
        \item We'll later see that the opposite direction is also correct, if restriction to all lines is low degree univariate, then \(f\) is low degree.
    \end{itemize}
\end{frame}
\begin{frame}
    \frametitle{Suggestion (or dead end?)}

    \begin{itemize}[<+->]
        \item It would be natural to suggest the following multivariate tester:
        \begin{enumerate}[<+->]
            \item Select a random line in \(\fcal^m\).
            \item Test that \(f\) restricted to the line is low degree univariate.
        \end{enumerate}
        \item Obviously, if \(f\) is low degree, the tester acceepts.
        \item But if \(f\) is \(\epsilon\)-far, how many lines will result in a rejection?
    \end{itemize}
\end{frame}
\begin{frame}
    \frametitle{Intuition - Summary}
    \begin{itemize}[<+->]
        \item Spoiler: The multivariate tester is exactly the tester we've just seen.
        \item We'll have to prove two main theorems.
        \item First, that in the univariate case, taking to a random line and testing the first \(d+2\) points has good rejection probability.
        \item Second, in the multivariate when reducing the test to the univariate case via a random line, then with good probability most lines will reject.
    \end{itemize}
\end{frame}

\section{Characterizations}
\begin{frame}
    \frametitle{Notation}
    \begin{itemize}
        \item \(L_{\bar{x},\bar{h}}\overset{\text{def}}{=}\braces{\bar{x}+i\bar{h}}_{i\in\fcal} \), a line parametrized by \(\bar{x},\bar{h}\).
        \begin{definition}
            The \alert{restriction of \(f\) to the line \(L_{\bar{x},\bar{h}}\)} is a the polynomial \(p\) such that \(p\of{i}=f\of{\bar{x}+i\bar{h}}\) for every \(i\in\fcal\).
        \end{definition}
    \end{itemize}
    \end{frame}
    \begin{frame}
        \frametitle{Local Characterization (Multivariate Polynomials)}
        \begin{theorem}
            Let \(\abs{F}>2d\). The function \(f:\fcal^m\to\fcal\) is in \(\pcal_{m,d}\) if and only if for every \(\bar{x},\bar{h} \in \fcal^m\) there exists a degree-\(d\) univariate polynomial \(p_{\bar{x},\bar{h}}\) such that \(p_{\bar{x},\bar{h}}=f\of{\bar{x}+i\bar{h}}\) for every \(i\in\fcal\).
        \end{theorem}
    \end{frame}

    \begin{frame}
        \frametitle{Proof \(\Rightarrow\)}
   \begin{proof}
    First direction is simple. For every \(\bar{x}=\of{x_1,\ldots,x_m}\) and \(\bar{h}=\of{h_1,\ldots,h_m}\) we have:
    \[
        f\of{\bar{x}+z\bar{h}}=f\of{x_1+zh_1,\ldots,x_m+zh_m}
    \]
    is univariate of degree \(d\)  in \(z\).
   \end{proof} 
    \end{frame}

    \begin{frame}
        \frametitle{Proof \(\Leftarrow\)}
        \begin{proof}
            The other direction is less trivial, we'll use the induction + probabilistic method.
            \begin{itemize}[<+->]
                \item Assume \(\deg\of{f}\leq 2d<\abs{\fcal}\) for now.
                \item For any all \(\bar{h}\in\fcal\), let \(g_{\bar{h}}\of{z}=f\of{z\bar{h}}\), the restriction of \(f\) to line \(L_{0,\bar{h}}\).
                \item \(g_{\bar{h}}\) is of degree \(\leq d\).
                \item The coefficient of \(z^{\deg\of{f}}\) is a (multivariate) polynomial in \(\bar{h}\) of degree \(\def\of{f}\leq 2d\), \(c\of{\bar{h}}\).
                \item Therefore, from Schwartz-Zippel lemma \(\ppr{\bar{h}\in \fcal^m}{c\of{\bar{h}}\neq0}>1-\frac{\deg\of{f}}{\abs{\fcal}}>0\).
                \item Since the probability over \(\bar{h}\) is positive, there exists such \(\bar{h}^*\) for which \(d \geq \deg\of{g_{\bar{h}^*}}=\deg\of{f}\).
            \end{itemize}
        \end{proof}
    \end{frame}
    
    \begin{frame}
        \frametitle{Proof \(\Leftarrow\)}
        \begin{proof}
        \begin{itemize}
            \item We now prove that \(\def\of{f}\leq 2d\).
            \item For each \(e\in\cbraces{0,\ldots,d}\) let \(f_e\of{x_1,\ldots,x_{m-1}}=f\of{x_1,\ldots,x_{m-1},e}\).
            \item \(f_e\of{\bar{x}}\) is an \(m-1\)-variate polynomial, by induction hypothesis (of the whole theorem), it is of degree \(\leq d\).
            \item For any \(e\in\cbraces{0,\ldots,d}\), let \(\delta_e\of{x}\) by a degree \(d\) polynomial such that \(\delta_e\of{e}=1\) and \(\delta_e\of{e'}=0\) for \(e'\in\cbraces{0,\ldots,d}\backslash \cbraces{e}\).
        \end{itemize}
        \end{proof}
    \end{frame}
    \begin{frame}
        \frametitle{Proof \(\Leftarrow\)}
        \begin{itemize}[<+->]
            \item For any \(e_1,\ldots,e_{m-1}\) it holds that:
             \[
                f\of{e_1,\ldots,e_{m-1},x} = \sum_{e=0}^{d} \delta_e\of{x}f_e\of{e_1,\ldots,e_{m-1}}
            \]
            \item Notice both sides are degree \(d\) polynomials that agree on \(d+1\) points: \(\cbraces{0,\ldots,d}\).
            \item Since the foregoing equality holds for all \(e_1,\ldots,e_{m-1}\) we conclude:
             \[
                f\of{x_1,\ldots,x_{m-1},x_m} = \sum_{e=0}^{d} \overbrace{\delta_e\of{x_m}}^{\deg \leq d}\overbrace{f_e\of{x_1,\ldots,x_{m-1}}}^{\deg \leq d}
            \]
        \end{itemize}
    \end{frame}

    \begin{frame}
        \frametitle{Conclusions}
        \begin{itemize}[<+->]
            \item  We've seen a ``local'' characterization of multivariate polynomial as a conjunction of \(\abs{\fcal}^{2m}\) constraints.
            \item However, these constraints are still not local, as the multivariate polynomial restricted to any line still refers to \(\abs{F}\) values of the tested function as a univariate polynomial.
            \item We will now give a local characterization of univariate polynomials as a conjunction of \(\abs{F}^2\) constraints, each referring only to \(d+2\) values of the function.
        \end{itemize}
    \end{frame}


    \begin{frame}
        \frametitle{A claim on deriviatives}
        \begin{itemize}[<+->]
            \item We start with a claim on \emph{deriviatives}.
            \begin{definition}
                For a function \(g:\fcal \to \fcal\), the \alert{deriviative} of \(g\) is the function \(g'\) such that: \(g'\of{x} \overset{\text{def}}{=}g\of{x+1}-g\of{x}\)
            \end{definition}
            \begin{theorem}
                A function \(g\) is of degree exactly \(d>0\) if and only if its deriviate is of degree exactly \(d-1\).
            \end{theorem}
        \end{itemize}
    \end{frame}
    \begin{frame}
        \frametitle{Proof about deriviatives}
        \begin{proof}
                Write \(g\of{x}=\sum_{j=0}^d c_jx^j\)
                \[
                    \begin{aligned}
                        g'\of{x} &= \sum_{j=0}^{d} c_j\cdot \of{x+1}^j - \sum_{j=0}^{d} c_j \cdot x^j &\\
                        &= \sum_{j=0}^d c_j \cdot \of{\of{x+1}^j - x^j} &\\
                        &= \sum_{j=0}^d c_j \cdot \sum_{k=0}^{j-1}{j \choose k} \cdot x^k & \Leftarrow \text{coefficient of } x^{d-1} \text{ is } c_d \cdot d \neq 0
                    \end{aligned}
                \]
        \end{proof}
    \end{frame}

    \begin{frame}
        \frametitle{Local Characterization - Univariate Polynomials}
        \begin{itemize}
            \item Notation: \(\alpha_i = \of{-1}^{i+1}\cdot {d+1 \choose i}\)
        \end{itemize}
        \begin{theorem}
            A univariate polynomial \(g:\fcal \to \fcal\) has degree \(\leq d<\abs{\fcal}\) if any only if for every \(e\in\fcal\) it holds that:
            \[
                \sum_{i=0}^{d+1}\alpha_i\cdot g\of{e+i} =0
            \]
        \end{theorem}
    \end{frame}

    \begin{frame}
        \frametitle{Proof}
        \only<1-4>{
        \begin{proof}
            \begin{itemize}[<+->]
                \item By induction on \(d\). If \(d=0\) the function is constant, therefore \(-g\of{e}+g\of{e+1}=0\) for all \(e\in\fcal\).
                \item For induction step, \(\deg\of{g}=d\) therefore \(\deg\of{g'}=d-1\)
                \item From induction hypothesis, since \(\deg\of{g'}=d-1\) then \(\sum_{i=0}^d \of{-1}^{i+1}\cdot {d \choose i}\cdot g'\of{e+i}=0\) for all \(e\in\fcal\).
                \item So we know that \(\deg\of{g}=d\) if and only if 
                \[
                    \sum_{i=0}^d \of{-1}^{i+1}\cdot {d \choose i}\cdot \of{g\of{e+i+1}-g\of{e+i}}=0
                \]
            \end{itemize}
        \end{proof}
        }
    \only<5->{
            \(
                \begin{aligned}
                    0 &= \sum_{i=0}^d \of{-1}^{i+1}\cdot {d \choose i}\cdot \of{g\of{e+i+1}-g\of{e+i}} \\
                    &= \sum_{i=0}^{d}\of{-1}^{i+1}\cdot {d\choose i}\cdot g\of{e+i+1} - \sum_{i=0}^{d}\of{-1}^{i+1}\cdot {d \choose i}\cdot g\of{e+i} \\
                    &= \sum_{j=1}^{d+1}\of{-1}^j \cdot {d \choose j-1 }\cdot g\of{e+j} - \sum_{i=0}^d\of{-1}^{i+1}\cdot {d\choose i}\cdot g\of{e+i} \\
                    &= g\of{e} + \of{-1}^{d+1}\cdot g\of{e+d+1}+\sum_{i=1}^{d}\of{-1}^{i}\cdot\of{{d \choose i-1}+ {d\choose i}}\cdot g\of{e+i} \\
                    &= -\sum_{i=0}^{d+1}\of{-1}^{i+1}{d+1 \choose i}\cdot g\of{e+i} \quad \square
                \end{aligned}
            \)
    }
    \end{frame}

\section{Tester}
\begin{frame}
    \frametitle{Tester}
    \begin{itemize}[<+->]
        \item The previous two characterizations result in the following tester:
        \begin{enumerate}[<+->]
            \item Sample uniformly \(\bar{x},\bar{h}\in \fcal^m\).
            \item Query \(f\) at \(\bar{x}, \bar{x}+\bar{h},\ldots,\bar{x}+\of{d+1}\bar{h}\).
            \item Accept if and only if \[
                \sum_{i=0}^{d+1}\alpha_i\cdot f\of{\bar{x}+i\bar{h}} = 0
            \]
        \end{enumerate}
        \item We now move to the analysis of this tester.
    \end{itemize}
\end{frame}

\begin{frame}
    \frametitle{Analysis - Main Theorem}
         \begin{theorem}
        Let \(\delta_0=1/\of{d+2}^2\). Then, our tester is a (one sided-error) POT with detection probability \(\min\of{\delta,\delta_0}/2\), where \(\delta\) denotes the distance of the given function from \(\pcal_{m,d}\)
    \end{theorem}
\end{frame}


\begin{frame}
    \frametitle{Main Theorem - Approach}
    \begin{itemize}[<+->]
        \item From previous characterizations if \(f\in \pcal_{m,d}\) we accept with probability 1.
        \item The approach to prove soundness is similar to the BLR theorem.
        \item We will show the if the tester rejects with probability \(\rho<\delta_0/2\) then \(f\), the input, is \(2\rho\)-close to \(\pcal_{m,d}\).
        \item This will be done by ``correcting'' \(f\) on at most \(2\rho\cdot \abs{\fcal}^m\) values into a low-degree polynomial.
    \end{itemize}
\end{frame}

\begin{frame}
    \frametitle{Correcting \(f\)}
    \begin{itemize}[<+->]
        \item The way \(f\) is corrected at point \(f\of{\bar{x}}\) into a low degree polynomial is according to the most frequent value of:
        \[
            \cbraces{\sum_{i\in[d+1]}\alpha_i\cdot f\of{\bar{x}+i\bar{h}}}_{\bar{h}\in \fcal^m}
        \]
        \item It remains to show that this ``corrected'' function is a low degree polynomial.
        \item And that this function is \(2\rho\)-close to \(f\).
    \end{itemize}
\end{frame}

\begin{frame}
    \frametitle{Simplified Version}
    \begin{itemize}[<+->]
        \item Imagine as if \(f\) is obtained by taking a degree \(\leq d\) polynomial \(p\) and ``corrupting'' it on less than \(\abs{\fcal}^m/2\of{d+1}\) points.
        \item What happens when \(f\) is fed to our tester?
        \item Case 1: \(\rho\geq\delta\), then we have a rejection probability proprtional to the distance.
    \end{itemize}
\end{frame}
\begin{frame}
    \frametitle{Simplified Version}
        \begin{itemize}[<+->]
        \item Case 2: \(\rho<\delta\) then the probability of \(\sum_{i\in[d+1]}\alpha_if\of{\bar{x}+i\bar{h}}=\sum_{i\in [d+1]}\alpha_ip\of{\bar{x}+i\bar{h}}\) is at least \(1-(d+1)\cdot\rho>1/2\), from union bound.
        \item Since the probability is greather than \(1/2\), then this is the most common value for \(f\), which means this is the value of the ``corrected'' \(f\).
        \item Since \(p\of{\bar{x}}\) is a polynomial of degree \(\leq d\), the right-hand-side equals \(p\of{\bar{x}}\).
        \end{itemize}
\end{frame}

\begin{frame}
    \frametitle{Back to Reality}

    \begin{itemize}[<+->]
        \item We can't assume that \(f\) is constructed that way.
        \item Instead, \(f\) satisfies:
        \[
            \ppr{\bar{x},\bar{h}\in \fcal^m}{\overbrace{\sum_{i=0}^{d+1}\alpha_i \cdot f\of{\bar{x}+i\bar{h}}=0}^{\triangle}} = 1-\rho
        \]
        \item For \(\rho<\delta_0/2 = 1/2\of{d+2}^2\).
        \item \(\triangle \iff f\of{\bar{x}}=\sum_{i\in [d+1]}\alpha_i\cdot f\of{\bar{x}+i\bar{h}}\)
    \end{itemize}
\end{frame}
\begin{frame}
    \frametitle{The Corrected Function}
    \begin{itemize}[<+->]
        \item Let \(S\) be a set.
        \item Let \(\mfo_{e\in S}\cbraces{v\of{e}}\) be the \emph{most frequently occuring} value of \(v\of{e}\), ties are broken arbitrarily.
        \item From a function \(f\) we construct a ``corrected'' function \(g\) as:
        \[
            g\of{\bar{x}} \overset{\text{def}}{=}\mfo_{\bar{h}\in\fcal^m}\cbraces{\sum_{i=1}^{d+1}\alpha_i\cdot f\of{\bar{x}+i\bar{h}}}
        \]
        \item From the assumption on \(f\), we get that \(g\of{\bar{x}}=f\of{\bar{x}}\) with good probability on a random \(\bar{x}\).
        \end{itemize}
\end{frame}
\begin{frame}
    \frametitle{Claims about \(g\)}
    \begin{itemize}[<+->]
        \item We'll prove the following claims on \(g\):
        \item Closeness: \(g\) is \(2\rho\)-close to \(f\).
        \item Strong Majority: 
        \[
            \forall \bar{x}\in \fcal^m: \ppr{\bar{h}\in\fcal^m}{g\of{\bar{x}}=\sum_{i=1}^{d+1}\alpha_i \cdot f\of{\bar{x}+i\bar{h}}} \geq 1-2\of{d+1}\rho
        \].
        \item Low Degree: \(g\in \pcal_{m,d}\).
    \end{itemize}
\end{frame}

\begin{frame}
    \frametitle{Closeness}
    \begin{theorem}
        The function \(g\) is \(2\rho\)-close to \(f\).
    \end{theorem}
    \begin{proof}
    \begin{itemize}[<+->]
        \item A point \(\bar{x}\) is \alert{bad} if \(g\of{\bar{x}}=\sum_{i=1}^{d+1}\alpha_i\cdot f\of{\bar{x}+i\bar{h}}\) for less than half of choices of \(\bar{h}\).
        \item A good point satisfies \(f\of{\bar{x}}=g\of{\bar{x}}\).
        \item Let \(B\) denote the set of all bad points.
    \end{itemize}
    \uncover<4->{
        \[ \begin{aligned}
            \rho &= \pr{\text{reject } f}\\
            &\geq \pr{\text{reject } f \mid \bar{x}\in B}\cdot  \pr{\bar{x}\in B}  \\
            &\geq \frac{1}{2} \cdot \pr{\bar{x} \in B}
        \end{aligned}
            \]
    }
        \pause{}
        \begin{itemize}
        \item \(\delta\of{f,g} = \ppr{\bar{x}\in\fcal^m}{f\of{\bar{x}} \neq g\of{\bar{x}}}\leq \pr{\bar{x} \in B} \leq 2\rho\).
        \end{itemize}
    \end{proof}
\end{frame}
\begin{frame}
    \frametitle{Strong Majority}

    \begin{theorem}
            \(\forall \bar{x}\in \fcal^m: \ppr{\bar{h}\in\fcal^m}{g\of{\bar{x}}=\sum_{i=1}^{d+1}\alpha_i \cdot f\of{\bar{x}+i\bar{h}}} \geq 1-2\of{d+1}\rho\)
    \end{theorem}
    \begin{proof}
        
    \only<1>{
        \begin{itemize}
            \item Let \(Z_{\bar{x}}=\sum_{i=1}^{d+1}\alpha_i\cdot f\of{\bar{x}+i\bar{h}}\) be a random variable where \(\bar{h}\in \fcal^m\) is chosen uniformly at random.
            \item The rejection probability is \(\rho\).
            \item Therefore: \(\ppr{\bar{x}\in \fcal^m}{f\of{\bar{x}}=Z_{\bar{x}}}=1-\rho\)
            \item This means that for a \alert{typical} \(\bar{x}\) we have \(Z_{\bar{x}}=f\of{\bar{x}}\) with high probability.
            \item Therefore, \(Z_{\bar{x}}=g\of{\bar{x}}\) for such \(\bar{x}\)'s as well W.H.P.
        \end{itemize}
    }
    \only<2>{
        \begin{itemize}
            \item However, we want to prove this claim for \alert{any} \(\bar{x}\).
            \item Therefore, we have to show that for \alert{all} \(\bar{x}\) the most common value of \(Z_{\bar{x}}\) is \textbf{very} common.
            \item From now on, fix some \(\bar{x}\in\fcal^m\)
        \end{itemize}
    }

    \only<3>{
\begin{itemize}
            \item We will use the property that \emph{the probability for the most common value to appear is lower bound by the collision probability}.
            \item Let \(S\) be a set and \(\mfo_S\) be the most frequent occuring value in \(S\).
            \item \(\overbrace{\ppr{u,v \in S}{u=v}}^{\text{collision probability}} = \sum_{v \in S}\ppr{u \in S}{u=v}^2 \leq \overbrace{\sum_{v\in S}\ppr{u\in S}{u=v}}^{\text{sum = 1}}\ppr{u\in S}{u=\mfo_S}=\ppr{u\in S}{u=\mfo_S}\)
            \item In our context \(S=\cbraces{\sum_{i=1}^{d+1}\alpha_i \cdot f\of{\bar{x}+i\bar{h}}\mid \bar{h}\in \fcal^m}\) and \(\mfo_S = g\of{\bar{x}}\).
\end{itemize}
    }
    \only<4>{
        \begin{itemize}
            \item So, to lower bound \(\pr{g\of{\bar{x}}= Z_{\bar{x}}}\) we will give a lower bound on the collision probability of \(Z_{\bar{x}}\) which is:
            \item \ppr{\bar{h}_1,\bar{h}_2\in\fcal^m}{\sum_{i=1}^{d+1}\alpha_i\cdot f\of{\bar{x}+i\bar{h}_1}=\sum_{j=1}^{d+1}\alpha_j \cdot f\of{\bar{x}+i\bar{h}_2}}
        \end{itemize}
    }
    \only<5>{
        \begin{itemize}
            \item Since \(\bar{h}_1,\bar{h_2}\) are random, then \(\bar{x}+i\bar{h}_1\) is also random and so is \(\bar{x}+j\bar{h}_2\).
            \item Therefore, we can apply our assumption (on \(\rho\) being the rej. prob. of \(f\)) on random line starting at random point in random direction.
            \item The random points will be \(\bar{x}+i\bar{h}_1\) for \(i\in [d+1]\) with direction \(\bar{h}_2\) and points \(\bar{x}+j\bar{h}_2\) with direction \(\bar{h}_1\).
            \item These lines intersect at points \(\bar{x}+i\bar{h}_1+j\bar{h}_2\), we will use these intersections soon.
        \end{itemize}
    }

    \only<6>{
        \begin{tikzpicture}

  \begin{scope}[scale=.4,shift={(0,0)}]
        \coordinate (Origin)   at (0,0);
        \coordinate (XAxisMin) at (-5,0);
        \coordinate (XAxisMax) at (5,0);
        \coordinate (YAxisMin) at (0,-5);
        \coordinate (YAxisMax) at (0,5);

        \begin{scope}
            \clip (-5,-5) rectangle (5,5); % Clips the picture...
            \pgftransformcm{0.7}{0.7}{-0.7}{0.7}{\pgfpoint{0cm}{0cm}}

            \foreach \x in {-15,...,15}
            \foreach \y in {-15,...,15}
            {
                \node[shape=circle,fill=black!25,scale=0.35] (\x-\y) at (\x,\y){};
            }
        \end{scope}

        \draw [dashed,thin, purple!55,->] (Origin) -- (5-0);
        \draw [dashed,thin, purple!55,->] (Origin) -- (0-5);
        \draw [dashed,thin, black!65] (3-0) -- (3-3);
        \draw [dashed,thin, black!65] (0-3) -- (3-3);

        \node[shape=circle,fill=purple,scale=0.35,label={[xshift=-0mm, yshift=-0mm]{\tiny \(\bar{x}+i\bar{h}_1+j\bar{h}_2\)}}] at (3-3){};

        \node[shape=circle,fill=black,scale=0.35,label={[yshift=-5mm]{\tiny \(\bar{x}\)}}] at (0-0){};
        \node[shape=circle,scale=0.35,label={[xshift=3mm,yshift=-3mm]{\tiny \(\bar{h}_1\)}}] at (5-0){};
        \node[shape=circle,fill=black,scale=0.35,label={[xshift=3mm,yshift=-5mm]{\tiny \(\bar{x}+i\bar{h}_1\)}}] at (3-0){};
        \node[shape=circle,,scale=0.35,label={[xshift=-3mm,yshift=-3mm]{\tiny \(\bar{h}_2\)}}] at (0-5){};
        \node[shape=circle,fill=black,scale=0.35,label={[xshift=-3mm,yshift=-5mm]{\tiny \(\bar{x}+j\bar{h}_2\)}}] at (0-3){};

    \end{scope}

  \end{tikzpicture}
    }

    \only<7>{
        \begin{itemize}
            \item For random \(\bar{h}_1,\bar{h}_2\) we have \(\bar{x}+i\bar{h}_1\) and \(\bar{x}+j\bar{h}_2\) are random for every \(i,j\in [d+1]\). This implies:
            \[
                \begin{aligned}
                   \forall i\in [d+1]: \ppr{\bar{h}_1,\bar{h}_2\in \fcal^m}{f\of{\bar{x}+i\bar{h}_1}=\sum_{j=1}^{d+1}\alpha_j\cdot f\of{\of{\bar{x}+i\bar{h}_1}+j\bar{h}_2}}=1-\rho \\
                   \forall j\in [d+1]: \ppr{\bar{h}_1,\bar{h}_2\in \fcal^m}{f\of{\bar{x}+j\bar{h}_2}=\sum_{i=1}^{d+1}\alpha_i\cdot f\of{\of{\bar{x}+j\bar{h}_2}+i\bar{h}_1}}=1-\rho \\
                \end{aligned}
            \]

        \end{itemize}
    }
    \only<8>{
        \begin{itemize}
            \item If we use a union bound over \(i\in [d+1]\) in the first eq. and over \(j\in [d+1]\) in the second one we get:
            \[
                \begin{aligned}
                   \ppr{\bar{h}_1,\bar{h}_2\in \fcal^m}{\sum_{i=1}^{d+1}\alpha_i f\of{\bar{x}+i\bar{h}_1}=\sum_{i=1}^{d+1}\sum_{j=1}^{d+1}\alpha_i\alpha_j\cdot f\of{\bar{x}+i\bar{h}_1+j\bar{h}_2}}=1-\of{d+1}\cdot\rho \\
                   \ppr{\bar{h}_1,\bar{h}_2\in \fcal^m}{\sum_{j=1}^{d+1}\alpha_j f\of{\bar{x}+j\bar{h}_2}=\sum_{j=1}^{d+1}\sum_{i=1}^{d+1}\alpha_j \alpha_i\cdot f\of{\bar{x}+i\bar{h}_1+j\bar{h}_2}}=1-\of{d+1}\cdot\rho \\
                \end{aligned}
            \]
        \end{itemize}
    }
    \only<9>{
        \begin{itemize}
            \item Union bounding over the previous two equations together, we get:
            \[
                   \underbrace{\ppr{\bar{h}_1,\bar{h}_2\in \fcal^m}{\overbrace{\sum_{i=1}^{d+1}\alpha_i f\of{\bar{x}+i\bar{h}_1}}^{\text{sample from }Z_{\bar{x}}}=\overbrace{\sum_{j=1}^{d+1}\alpha_j f\of{\bar{x}+j\bar{h}_2}}^{\text{sample from }Z_{\bar{x}}}}}_{\text{collision probability of }Z_{\bar{x}}}=1-2\of{d+1}\cdot\rho
            \]
        \end{itemize}
    }
    \only<10>{
        \begin{itemize}
            \item We have given a lower bound on the collision probability of \(Z_{\bar{x}}\).
            \item This is a lower bound on \(\ppr{\bar{h}\in\fcal^m}{g\of{\bar{x}}=\overbrace{\sum_{i=1}^{d+1}\alpha_i\cdot f\of{\bar{x}+i\bar{h}}}^{=\text{sample from }Z_{\bar{x}}}}\)
            \item Finally, QED.
        \end{itemize}
    }
    \end{proof}
\end{frame}
\begin{frame}
    \frametitle{Low Degree}
    \begin{theorem}
        \(g\in \pcal_{m,d}\).
    \end{theorem}
    \begin{proof}
        \only<1>{
        \begin{itemize}
            \item \(g\) is low degree \(\iff\) for all \(\bar{x},\bar{h}\in \fcal^m\): \(\sum_{i=0}^{d+1}\alpha_i\cdot g\of{\bar{x}+i\bar{h}}=0\).
            \item Fix \(\bar{x},\bar{h}\) for the rest of the proof.
        \end{itemize}
        }
        \only<2>{
            \begin{itemize}
                \item Let \(\bar{h}_1,\bar{h}_2\) be uniform and independently chosen from \(\fcal^m\).
                \item Therefore, for every \(i\in [d+1]\) \(\bar{h}_1+i\bar{h}_2\) is uniform in \(\fcal^m\) and from ``Strong Majority'' lemma:
                \[
                    \ppr{\bar{h}_1,\bar{h}_2\in\fcal^m}{g\of{\bar{x}+i\bar{h}}=\sum_{j=1}^{d+1}\alpha_j\cdot f\of{\of{\bar{x}+i\bar{h}}+j\of{\bar{h}_1+i\bar{h}_2}}}\geq 1-2\of{d+1}\rho
                \]
            \end{itemize}
        }
        \only<3>{
            \begin{itemize}
                \item Also, for all \(j\in [d+1]\) both \(\bar{x}+j\bar{h}_1\) and \(\bar{h}+j\bar{h}_2\) are uniformly distributed in \(\fcal^m\).
                \item Therefore, from \(\rho\) being the rejection probability of the tester for \(f\) we get:
                \[
                    \ppr{\bar{h}_1,\bar{h}_2\in\fcal^m}{\sum_{i=0}^{d+1}\alpha_i f\of{\of{\bar{x}+j\bar{h}_1}+i\of{\bar{h}+j\bar{h}_2}}=0} = 1 - \rho
                \]
            \end{itemize}
        }
        \only<4>{
            \begin{itemize}
                \item By adding up the above equations for all \(j\in [d+1]\) and rephrasing the agument to \(f\), we get:
                \[
                    \ppr{\bar{h}_1,\bar{h}_2\in\fcal^m}{\sum_{j=1}^{d+1}\alpha_j \sum_{i=0}^{d+1}\alpha_i f\of{\of{\bar{x}+i\bar{h}}+j\of{\bar{h}_1+i\bar{h}_2}}=0} = 1 - \of{d+1}\rho
                \]
                \item Overall, from union bounding, we get...
            \end{itemize}
        }
        \only<5>{
            \[
                \begin{aligned}
                   &\ppr{\bar{h}_1,\bar{h}_2\in\fcal^m}{\sum_{i=0}^{d+1}\alpha_i g\of{\bar{x}+i\bar{h}}=\sum_{i=0}^{d+1}\alpha_i \sum_{j=1}^{d+1}\alpha_j f\of{\of{\bar{x}+i\bar{h}}+j\of{\bar{h}_1+i\bar{h}_2}}=0}  \\
                    \geq & 1-\of{d+2}\cdot 2\of{d+1}\rho - \of{d+1}\cdot \rho \\
                    = & 1-\of{2d +5}\of{d+1}\rho \\
                    > & 0 \quad \texttt{//since } \rho\geq 1/2\of{d+2}^2
                \end{aligned}
            \]
        }
        \only<6>{
            \begin{itemize}
                \item There we get that:
                \[
                    \ppr{\bar{h}_1,\bar{h}_2\in \fcal^m}{\sum_{i=0}^{d+1}\alpha_i g\of{\bar{x}+i\bar{h}}=0} > 0
                \]
                \item Since the event is fixed (ind. of \(\bar{h}_1,\bar{h}_2\)), the probability is therefore 1.
            \end{itemize}
        }
    \end{proof}

    

\end{frame}
    \begin{frame}
        \frametitle{Recap}
    \begin{itemize}
        \item We have shown that \(f\) is \(2\rho\)-far from \(g\) if the detection probability is below some constant threshold.
        \item We have shown that \(g\) in that case is a low degree polynomial.
        \item The tester doesn't depend on the distance.
        \item The tester always accepts low degree polynomials.
        \item Therefore: We have a one-sided error POT for low degree polynomials.
    \end{itemize}
\end{frame}
\begin{frame}
    \frametitle{Fin.}

    \Large Thank you!
    

\end{frame}

\end{document}